\documentclass[b5paper,11pt,oneside,fleqn]{article}
\usepackage{amsmath,amsthm,amssymb}

\usepackage{geometry}

\title{\raggedleft\sffamily\Huge fluid}
\author{\sffamily\itshape Bacco, Giacomo}
\date{}

\begin{document}
\maketitle

\section{Flow past a cylinder}

Let $ \rho_0 $ be the radius of the cylinder, $ \rho,\xi $ the polar coordinate 
system in use.
One possible solution of this problem have these potential and streamline 
functions:
\begin{align}
\phi(\rho,\xi) &= \left( \rho + \frac{\rho_0^2}{\rho} \right) \cos\xi \\[1ex]
\psi(\rho,\xi) &= \left( \rho - \frac{\rho_0^2}{\rho} \right) \sin\xi 
\end{align}
Although these equations are deeply coupled, the radius $ \rho $ and the phase 
$ \xi $ can be obtained as a function of the other quantities.
For our purposes, we use $ \psi $.
\begin{align}
\rho(\psi,\xi) &= \frac{\psi + \sqrt{\psi^2 + 4\rho_0^2 \sin^2\xi}}{2\sin\xi} 
\\[1ex]
\xi(\psi,\rho) &= \arcsin \left( \frac{\rho\, \psi}{\rho^2 - \rho_0^2} \right)  
\end{align}




\end{document}